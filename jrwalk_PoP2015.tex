\documentclass[12pt,floatfix,showpacs]{revtex4-1}

\usepackage{amssymb,amsmath,amsfonts}
\usepackage{graphicx}

\newcommand{\eg}{\emph{e.g., }}
\newcommand{\ie}{\emph{i.e., }}
\newcommand{\etal}{\emph{et al.}}

% remove these for final publication 
\usepackage{color} 
\newcommand{\note}[1]{\textcolor{red}{#1}}
\newcommand{\gnote}[1]{\marginpar{\textcolor{red}{\scriptsize{#1}}}}

\graphicspath{{./graphics/}{./pdfgraphics/}}  % remove pdfgraphics path for final publication!

\begin{document}

\title{Stability and Edge-Localized Mode Characterization in I-Mode Pedestals}

\author{JR Walk}
\email[]{jrwalk@psfc.mit.edu}
\affiliation{MIT Plasma Science and Fusion Center}

\author{JW Hughes}
\affiliation{MIT Plasma Science and Fusion Center}

\author{PB Snyder}
\affiliation{General Atomics}

\author{AE Hubbard}
\affiliation{MIT Plasma Science and Fusion Center}

\author{B LaBombard}
\affiliation{MIT Plasma Science and Fusion Center}

\author{DF Brunner}
\affiliation{MIT Plasma Science and Fusion Center}

\author{JL Terry}
\affiliation{MIT Plasma Science and Fusion Center}

\author{DG Whyte}
\affiliation{MIT Plasma Science and Fusion Center}

\author{AE White}
\affiliation{MIT Plasma Science and Fusion Center}

\author{E Edlund}
\affiliation{Princeton Plasma Physics Laboratory}

\date{\today}

\begin{abstract}
I-mode is a novel high-confinement tokamak regime characterized by H-mode-like enhanced energy confinement and the formation of a strong temperature pedestal, without the accompanying density pedestal or enhanced particle confinement, maintaining an L-mode-like density profile.  I-mode exhibits a number of desirable properties for a reactor regime, including a lack of strong degradation of energy confinement with heating power and apparent naturally-occurring suppression of large ELMs, avoiding the need for externally-applied ELM suppression.  However, under certain conditions (particularly, reduced toroidal field) small, intermittent ELM-like events are seen, although these cases are modeled to be stable to the peeling-ballooning MHD instability associated with the ELM trigger, as is typical of I-mode pedestals.  We examine these events in detail to better characterize the edge stability behavior in I-mode.  The majority of observed ELM candidates are observed to be synchronized with the sawtooth heat pulse reaching the pedestal, which measurably perturbs the temperature pedestal.  However, this perturbation appears to be insufficient to reach the peeling-ballooning stability boundary; moreover, the ELM candidate does not include a ``crash'' in the pedestal temperature or stored energy.  \note{precursor fluctuations?  include Ahmed?  Ref divertor heat flux measurements?}  In short, these events do not appear to be true instability-driven ELMs, but rather are benign $H_\alpha$ spikes driven by the sawtooth heat pulse.  A minority of the ELM candidates in I-mode do include the characteristic temperature crash associated with an ELM, and are not necessarily sawtooth-triggered -- however, these events are isolated, and the stationary pedestal structure in these I-modes is also modeled to be stable to the ELM trigger, indicating that transient events in the pedestal drive these ELMs, rather than an inherent instability of the pedestal.
\end{abstract}

\pacs{52.55.Fa, 52.55.Tn, 52.35.Py, 52.25.Fi, 52.40.Hf}

\maketitle

%%%%%%%%%%%%%%%%%%%%%%%%%%%%%%%%%%%%%%%%%%%%%%%%%%%%%%

\section{Introduction}\label{sec:intro}

The development of tokamak magnetic-confinement fusion into a viable \& economical form of power generation faces two overarching (and seemingly contradictory) requirements.  
First, a high level of energy confinement is necessary for net energy production with the desired level of self-heating of the plasma by fusion products.  
At the same time, sufficient particle transport is needed to avoid the deleterious effects of accumulated impurities (both helium ``fusion ash'' and higher-$Z$ impurities from the erosion of plasma-facing components) due to fuel dilution and radiative losses.  
This has been achieved in a number of operating regimes, collectively termed ``high-confinement'' or H-modes \cite{Wagner1982}.  

H-modes are characterized by a steep gradient region at the plasma edge in density, temperature, and pressure, termed the \emph{pedestal}, the height of which is strongly correlated with global fusion performance \cite{Kinsey2011}.  
These strong gradients have been shown to drive edge MHD instabilities \cite{Huysmans2005,Maget2013,Snyder2002} resulting in an Edge-Localized Mode (ELM), an explosive perturbation to the pedestal expelling energy and particles into the plasma exhaust \cite{Zohm1996}.  
On existing experiments, ELMs drive sufficient particle transport to allow stationary operation with acceptable radiative losses \cite{Keilhacker1984}; as ELMy H-modes are robust and straightforward to achieve, this regime is considered the baseline for ITER operation \cite{ITER1999,Shimada2007}.  
However, on ITER-scale devices ELMs drive transient heat loads to the divertor, leading to unacceptable levels of erosion and damage to plasma-facing components \cite{Loarte2003,Federici2003}.
This introduces an additional requirement for tokamak fusion reactor concepts -- the avoidance, suppression, or mitigation of large ELMs, either via externally-applied engineering solutions (pellet pacing \cite{Baylor2013,Lang2014} or resonant magnetic perturbations \cite{Evans2004,Evans2006}), or via alternate high-confinement regimes which regulate the pedestal below the ELM limit (\eg the Enhanced $D_\alpha$ (EDA) H-mode \cite{Greenwald1999,Hubbard2001} or the QH-mode \cite{Burrell2002,Suttrop2005}).

\begin{figure}[p] % shift to ht placement!
 \includegraphics[width=\textwidth]{trace_imode.pdf}
 \caption{(left) characteristic time traces for an I-mode.  After the L-I transition, the core and edge temperature rise over several sawtooth cycles (visible in the oscillations in $T_e(0)$ and $T_{e,ped}$) before reaching a steady level; global pressure and confinement rise accordingly.  However, the density remains unchanged from the L-mode level.  No ELMs are exhibited on the $D_\alpha$ trace.  (right) pedestal profiles for L-, I-, and H-modes.  The I-mode (green) retains a density profile similar to L-mode (black), unlike the ELMy (red) and EDA (blue) H-modes, which form a strong density pedestal.  However, the I-mode forms a higher temperature pedestal than either H-mode, resulting in comparable pedestal pressures to H-mode while retaining L-mode particle transport.}
 \label{fig:imode_trace}
\end{figure}

The I-mode \cite{Whyte2010,Hubbard2011,Walk2014,Walk2014b}, pioneered on the Alcator C-Mod tokamak \cite{Hutchinson1994}, is one such alternate regime for high-performance operation.  
I-mode is notably unique among high-performance regimes in that it apparently decouples energy and particle transport, attaining the desired H-mode-like energy confinement while retaining L-mode particle transport, naturally achieving the desired low level of impurity confinement \cite{Howard2011}.\note{other cites?}  
This manifests in the edge by the formation of a strong temperature pedestal without the accompanying density pedestal found in conventional H-modes (see Figure~\ref{fig:imode_trace}).  
I-mode also exhibits minimal degradation of energy confinement with heating power \cite{Whyte2010,Dominguez2012,Walk2014b}, in contrast to the degradation of confinement found in ELMy H-mode (roughly $\tau_E \sim P^{-0.7}$ from multi-machine analyses \cite{Christiansen1992,ITER1999}), a potentially highly favorable result for a reactor regime.

The I-mode appears to generally lack large ELMs, obviating the need for externally-applied engineering solutions; however, as has been previously reported \cite{Whyte2010,Walk2014b}\note{check citations!}, under certain conditions (particularly at reduced toroidal field) small, intermittent ELMs have been observed.
In this paper, we examine these intermittent events through computational modeling of the instabilities associated with the ELM trigger and experimental observations of edge behavior in I-mode.  \note{sort out outline}

\section{I-Mode Access \& Experimental Setup}\label{sec:setup}

The I-mode experiments presented here were carried out on the Alcator C-Mod tokamak \cite{Hutchinson1994}, a compact, high-field device with major radius $R \sim 0.67 \;\mbox{m}$, minor radius $a \sim 0.22 \;\mbox{m}$, and toroidal field up to $B_T \le 8.1 \;\mbox{T}$.  
C-Mod operates with entirely high-$Z$ metal plasma-facing components, and reaches comparable heat flux to the divertor to that expected for ITER \cite{Loarte2007,Terry2007,LaBombard2011}, making it a uniquely-suited test bed for SOL and divertor experiments in support of ITER\gnote{too much?}.  
High-performance operation is commonly assisted by boronization treatment of plasma-facing materials -- however, due to the low impurity confinement in I-mode a recent boronization is not critical for these experiments.  
Alcator C-Mod plasmas are purely RF heated with up to $5.5 \;\mbox{MW}$ of ion-cyclotron heating power.

I-mode operation is most attainable in the so-called ``unfavorable'' drift configuration -- that is, with the ion $\nabla B$ drift directed away from the primary X-point \cite{Whyte2010}.  
This elevates the H-mode power threshold \cite{Hubbard2007}, widening the power range between L- and H-mode available for I-mode operation.  
This drift configuration both in upper-single-null (USN) shapes, and in the more typical lower-single-null (LSN) shapes with the toroidal field reversed (plasma current direction is reversed as well to preserve field helicity)\gnote{clarify?}.  
Brief I-mode periods have also been observed in the favorable drift configuration, but these transitioned quickly to H-mode and are not considered for the purposes of this paper.

Provided the unfavorable drift condition is met, I-mode operation is quite robustly accessible on Alcator C-mod, with steady I-modes attained in a number of shapes and edge current profiles -- notably, I-mode operates naturally near the values for edge safety factor and collisionality targeted for ITER.  
The data presented here were taken in dedicated I-mode experiments focusing on reversed-field LSN operation to optimize the plasma for pedestal and edge diagnostic coverage and high-quality pedestal profiles for modeling purposes.  
In a subset of these experiments, small, intermittent ELMs were observed -- while the conditions leading to these events are not well-understood, they are particularly found in operation at reduced field ($B_T \sim 4.6 \;\mbox{T}$).  
These results are the focus of this paper, with broader results from the pedestal study reported by Walk \emph{et al.} \cite{Walk2014}.\gnote{get formatting right}

\section{I-Mode Pedestal Stability \& the EPED Model}\label{sec:model}

Large, uncontrolled ELMs in ITER operation are expected to drive unacceptable levels of pulsed heat loading and erosion damage to wall and divertor surfaces \cite{Loarte2003,Federici2003} -- as such, avoiding, mitigating, or suppressing large ELMs is a major focus in research on high-performance regimes.  Confidence in plans for high-performance operation on ITER- and reactor-scale devices without large ELMs requires a predictive, first-principles understanding of the pedestal structure and stability at the ELM limit.  Recent cooperative efforts among theory, modeling, and experiment \cite{Groebner2013} has resulted in such a model, termed EPED (not an acronym) \cite{Snyder2009,Snyder2011}.  The EPED model combines constraints from stability against peeling-ballooning MHD modes driven by the steep gradients in the pedestal

\begin{figure}[ht]
 \includegraphics[width=\textwidth]{pdfgraphics/1120824019_ELITE_stitch_v2.pdf}
 \caption{MHD stability contour for high-current ($1.3 \;\mbox{MA}$), high-performance I-mode generated by the ELITE code.  The experimental measurement (with uncertainties) is shown by the crosshair, with the stability boundary (accounting for diamagnetic stabilization of the MHD mode) shown by the yellow dashed line.  Experimentally-measured parameters for the modeled phase are shown at right.  The white contours indicate calculations of the infinite-$n$ ballooning stability calculated by BALOO, used as a surrogate for the onset of KBM turbulence.  Due to the local nature of the infinite-n constraint, BALOO calculates the width in flux space that is locally ballooning-critical.  When this reaches half of the pedestal width, the KBM is assumed to be triggered. This case is near the KBM-predicted pedestal width $\Delta_{EPED}$, but is nevertheless modeled to be KBM-stable (for $\Delta_\psi \sim 0.02$, the half-width threshold is the white dotted contour labeled $0.01$.}
 \label{fig:elite_1120824019}
\end{figure}

\begin{figure}[ht]
 \includegraphics[width=0.75\textwidth]{pdfgraphics/prof_elmy_imode.pdf}
 \caption{Pedestal profiles in I-mode and ELMy H-mode. Due to the steep density gradient in the pedestal, the H-mode exhibits significant pressure gradient and edge current density, which drive the peeling-ballooning MHD instability associated with the ELM trigger. Despite this, the high edge temperature in I-mode allows it to reach an appreciable pedestal pressure.}
 \label{fig:prof_elmy_imode}
\end{figure}

%%%%%%%%%%%%%%%%%%%%%%%%%%%%%%%%%%%%%%%%%%%%%%%%%%%%%%

\begin{acknowledgments}
 Experimental work on Alcator C-Mod is supported by US DOE agreement DE-FC02-99ER54512. Theory work at General Atomics is supported by US DOE agreement DE-FG02-99ER54309.  The authors also wish to acknowledge the efforts of the Alcator C-Mod team for supporting the experiments reported here.
\end{acknowledgments}

% remove \bibliography call, paste in bbl file
% actually, peerx-press says they accept bib files now!
% construct proper bib file once citations are all put together
\bibliographystyle{aipnum4-1}
\bibliography{jrwalk_references}

\end{document}