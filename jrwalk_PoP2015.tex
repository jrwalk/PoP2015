\documentclass[12pt,floatfix,showpacs]{revtex4-1}

\usepackage{amssymb,amsmath,amsfonts}
\usepackage{graphicx}

\graphicspath{{./graphics/}}

\begin{document}

\title{Stability and Edge-Localized Mode Characterization in I-Mode Pedestals}

\author{JR Walk}
\email[]{jrwalk@psfc.mit.edu}
\affiliation{MIT Plasma Science and Fusion Center}

\author{JW Hughes}
\affiliation{MIT Plasma Science and Fusion Center}

\author{PB Snyder}
\affiliation{General Atomics}

\author{AE Hubbard}
\affiliation{MIT Plasma Science and Fusion Center}

\author{B LaBombard}
\affiliation{MIT Plasma Science and Fusion Center}

\author{DF Brunner}
\affiliation{MIT Plasma Science and Fusion Center}

\author{JL Terry}
\affiliation{MIT Plasma Science and Fusion Center}

\author{DG Whyte}
\affiliation{MIT Plasma Science and Fusion Center}

\author{AE White}
\affiliation{MIT Plasma Science and Fusion Center}

\date{\today}

\begin{abstract}
I-mode is a novel high-confinement tokamak regime characterized by H-mode-like enhanced energy confinement and the formation of a strong temperature pedestal, without the accompanying density pedestal or enhanced particle confinement, maintaining an L-mode-like density profile.  I-mode exhibits a number of desirable properties for a reactor regime, including a lack of strong degradation of energy confinement with heating power and apparent naturally-occurring suppression of large ELMs, avoiding the need for externally-applied ELM suppression.
\end{abstract}

\pacs{52.55.Fa, 52.55.Tn, 52.35.Py, 52.25.Fi, 52.40.Hf}

\maketitle

%%%%%%%%%%%%%%%%%%%%%%%%%%%%%%%%%%%%%%%%%%%%%%%%%%%%%%

\section{Introduction}\label{sec:intro}

%%%%%%%%%%%%%%%%%%%%%%%%%%%%%%%%%%%%%%%%%%%%%%%%%%%%%%

\begin{acknowledgments}
 Experimental work on Alcator C-Mod is supported by US DOE agreement DE-FC02-99ER54512. Theory work at General Atomics is supported by US DOE agreement DE-FG02-99ER54309.  The authors also wish to acknowledge the efforts of the Alcator C-Mod group for supporting the experiments reported here.
\end{acknowledgments}

\bibliographystyle{aipnum4-1}
\bibliography{jrwalk_references}

\end{document}